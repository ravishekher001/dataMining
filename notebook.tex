
% Default to the notebook output style

    


% Inherit from the specified cell style.




    
\documentclass[11pt]{article}

    
    
    \usepackage[T1]{fontenc}
    % Nicer default font (+ math font) than Computer Modern for most use cases
    \usepackage{mathpazo}

    % Basic figure setup, for now with no caption control since it's done
    % automatically by Pandoc (which extracts ![](path) syntax from Markdown).
    \usepackage{graphicx}
    % We will generate all images so they have a width \maxwidth. This means
    % that they will get their normal width if they fit onto the page, but
    % are scaled down if they would overflow the margins.
    \makeatletter
    \def\maxwidth{\ifdim\Gin@nat@width>\linewidth\linewidth
    \else\Gin@nat@width\fi}
    \makeatother
    \let\Oldincludegraphics\includegraphics
    % Set max figure width to be 80% of text width, for now hardcoded.
    \renewcommand{\includegraphics}[1]{\Oldincludegraphics[width=.8\maxwidth]{#1}}
    % Ensure that by default, figures have no caption (until we provide a
    % proper Figure object with a Caption API and a way to capture that
    % in the conversion process - todo).
    \usepackage{caption}
    \DeclareCaptionLabelFormat{nolabel}{}
    \captionsetup{labelformat=nolabel}

    \usepackage{adjustbox} % Used to constrain images to a maximum size 
    \usepackage{xcolor} % Allow colors to be defined
    \usepackage{enumerate} % Needed for markdown enumerations to work
    \usepackage{geometry} % Used to adjust the document margins
    \usepackage{amsmath} % Equations
    \usepackage{amssymb} % Equations
    \usepackage{textcomp} % defines textquotesingle
    % Hack from http://tex.stackexchange.com/a/47451/13684:
    \AtBeginDocument{%
        \def\PYZsq{\textquotesingle}% Upright quotes in Pygmentized code
    }
    \usepackage{upquote} % Upright quotes for verbatim code
    \usepackage{eurosym} % defines \euro
    \usepackage[mathletters]{ucs} % Extended unicode (utf-8) support
    \usepackage[utf8x]{inputenc} % Allow utf-8 characters in the tex document
    \usepackage{fancyvrb} % verbatim replacement that allows latex
    \usepackage{grffile} % extends the file name processing of package graphics 
                         % to support a larger range 
    % The hyperref package gives us a pdf with properly built
    % internal navigation ('pdf bookmarks' for the table of contents,
    % internal cross-reference links, web links for URLs, etc.)
    \usepackage{hyperref}
    \usepackage{longtable} % longtable support required by pandoc >1.10
    \usepackage{booktabs}  % table support for pandoc > 1.12.2
    \usepackage[inline]{enumitem} % IRkernel/repr support (it uses the enumerate* environment)
    \usepackage[normalem]{ulem} % ulem is needed to support strikethroughs (\sout)
                                % normalem makes italics be italics, not underlines
    

    
    
    % Colors for the hyperref package
    \definecolor{urlcolor}{rgb}{0,.145,.698}
    \definecolor{linkcolor}{rgb}{.71,0.21,0.01}
    \definecolor{citecolor}{rgb}{.12,.54,.11}

    % ANSI colors
    \definecolor{ansi-black}{HTML}{3E424D}
    \definecolor{ansi-black-intense}{HTML}{282C36}
    \definecolor{ansi-red}{HTML}{E75C58}
    \definecolor{ansi-red-intense}{HTML}{B22B31}
    \definecolor{ansi-green}{HTML}{00A250}
    \definecolor{ansi-green-intense}{HTML}{007427}
    \definecolor{ansi-yellow}{HTML}{DDB62B}
    \definecolor{ansi-yellow-intense}{HTML}{B27D12}
    \definecolor{ansi-blue}{HTML}{208FFB}
    \definecolor{ansi-blue-intense}{HTML}{0065CA}
    \definecolor{ansi-magenta}{HTML}{D160C4}
    \definecolor{ansi-magenta-intense}{HTML}{A03196}
    \definecolor{ansi-cyan}{HTML}{60C6C8}
    \definecolor{ansi-cyan-intense}{HTML}{258F8F}
    \definecolor{ansi-white}{HTML}{C5C1B4}
    \definecolor{ansi-white-intense}{HTML}{A1A6B2}

    % commands and environments needed by pandoc snippets
    % extracted from the output of `pandoc -s`
    \providecommand{\tightlist}{%
      \setlength{\itemsep}{0pt}\setlength{\parskip}{0pt}}
    \DefineVerbatimEnvironment{Highlighting}{Verbatim}{commandchars=\\\{\}}
    % Add ',fontsize=\small' for more characters per line
    \newenvironment{Shaded}{}{}
    \newcommand{\KeywordTok}[1]{\textcolor[rgb]{0.00,0.44,0.13}{\textbf{{#1}}}}
    \newcommand{\DataTypeTok}[1]{\textcolor[rgb]{0.56,0.13,0.00}{{#1}}}
    \newcommand{\DecValTok}[1]{\textcolor[rgb]{0.25,0.63,0.44}{{#1}}}
    \newcommand{\BaseNTok}[1]{\textcolor[rgb]{0.25,0.63,0.44}{{#1}}}
    \newcommand{\FloatTok}[1]{\textcolor[rgb]{0.25,0.63,0.44}{{#1}}}
    \newcommand{\CharTok}[1]{\textcolor[rgb]{0.25,0.44,0.63}{{#1}}}
    \newcommand{\StringTok}[1]{\textcolor[rgb]{0.25,0.44,0.63}{{#1}}}
    \newcommand{\CommentTok}[1]{\textcolor[rgb]{0.38,0.63,0.69}{\textit{{#1}}}}
    \newcommand{\OtherTok}[1]{\textcolor[rgb]{0.00,0.44,0.13}{{#1}}}
    \newcommand{\AlertTok}[1]{\textcolor[rgb]{1.00,0.00,0.00}{\textbf{{#1}}}}
    \newcommand{\FunctionTok}[1]{\textcolor[rgb]{0.02,0.16,0.49}{{#1}}}
    \newcommand{\RegionMarkerTok}[1]{{#1}}
    \newcommand{\ErrorTok}[1]{\textcolor[rgb]{1.00,0.00,0.00}{\textbf{{#1}}}}
    \newcommand{\NormalTok}[1]{{#1}}
    
    % Additional commands for more recent versions of Pandoc
    \newcommand{\ConstantTok}[1]{\textcolor[rgb]{0.53,0.00,0.00}{{#1}}}
    \newcommand{\SpecialCharTok}[1]{\textcolor[rgb]{0.25,0.44,0.63}{{#1}}}
    \newcommand{\VerbatimStringTok}[1]{\textcolor[rgb]{0.25,0.44,0.63}{{#1}}}
    \newcommand{\SpecialStringTok}[1]{\textcolor[rgb]{0.73,0.40,0.53}{{#1}}}
    \newcommand{\ImportTok}[1]{{#1}}
    \newcommand{\DocumentationTok}[1]{\textcolor[rgb]{0.73,0.13,0.13}{\textit{{#1}}}}
    \newcommand{\AnnotationTok}[1]{\textcolor[rgb]{0.38,0.63,0.69}{\textbf{\textit{{#1}}}}}
    \newcommand{\CommentVarTok}[1]{\textcolor[rgb]{0.38,0.63,0.69}{\textbf{\textit{{#1}}}}}
    \newcommand{\VariableTok}[1]{\textcolor[rgb]{0.10,0.09,0.49}{{#1}}}
    \newcommand{\ControlFlowTok}[1]{\textcolor[rgb]{0.00,0.44,0.13}{\textbf{{#1}}}}
    \newcommand{\OperatorTok}[1]{\textcolor[rgb]{0.40,0.40,0.40}{{#1}}}
    \newcommand{\BuiltInTok}[1]{{#1}}
    \newcommand{\ExtensionTok}[1]{{#1}}
    \newcommand{\PreprocessorTok}[1]{\textcolor[rgb]{0.74,0.48,0.00}{{#1}}}
    \newcommand{\AttributeTok}[1]{\textcolor[rgb]{0.49,0.56,0.16}{{#1}}}
    \newcommand{\InformationTok}[1]{\textcolor[rgb]{0.38,0.63,0.69}{\textbf{\textit{{#1}}}}}
    \newcommand{\WarningTok}[1]{\textcolor[rgb]{0.38,0.63,0.69}{\textbf{\textit{{#1}}}}}
    
    
    % Define a nice break command that doesn't care if a line doesn't already
    % exist.
    \def\br{\hspace*{\fill} \\* }
    % Math Jax compatability definitions
    \def\gt{>}
    \def\lt{<}
    % Document parameters
    \title{CaseStudy8\_CherryBlossomData\_Finalv1}
    
    
    

    % Pygments definitions
    
\makeatletter
\def\PY@reset{\let\PY@it=\relax \let\PY@bf=\relax%
    \let\PY@ul=\relax \let\PY@tc=\relax%
    \let\PY@bc=\relax \let\PY@ff=\relax}
\def\PY@tok#1{\csname PY@tok@#1\endcsname}
\def\PY@toks#1+{\ifx\relax#1\empty\else%
    \PY@tok{#1}\expandafter\PY@toks\fi}
\def\PY@do#1{\PY@bc{\PY@tc{\PY@ul{%
    \PY@it{\PY@bf{\PY@ff{#1}}}}}}}
\def\PY#1#2{\PY@reset\PY@toks#1+\relax+\PY@do{#2}}

\expandafter\def\csname PY@tok@w\endcsname{\def\PY@tc##1{\textcolor[rgb]{0.73,0.73,0.73}{##1}}}
\expandafter\def\csname PY@tok@c\endcsname{\let\PY@it=\textit\def\PY@tc##1{\textcolor[rgb]{0.25,0.50,0.50}{##1}}}
\expandafter\def\csname PY@tok@cp\endcsname{\def\PY@tc##1{\textcolor[rgb]{0.74,0.48,0.00}{##1}}}
\expandafter\def\csname PY@tok@k\endcsname{\let\PY@bf=\textbf\def\PY@tc##1{\textcolor[rgb]{0.00,0.50,0.00}{##1}}}
\expandafter\def\csname PY@tok@kp\endcsname{\def\PY@tc##1{\textcolor[rgb]{0.00,0.50,0.00}{##1}}}
\expandafter\def\csname PY@tok@kt\endcsname{\def\PY@tc##1{\textcolor[rgb]{0.69,0.00,0.25}{##1}}}
\expandafter\def\csname PY@tok@o\endcsname{\def\PY@tc##1{\textcolor[rgb]{0.40,0.40,0.40}{##1}}}
\expandafter\def\csname PY@tok@ow\endcsname{\let\PY@bf=\textbf\def\PY@tc##1{\textcolor[rgb]{0.67,0.13,1.00}{##1}}}
\expandafter\def\csname PY@tok@nb\endcsname{\def\PY@tc##1{\textcolor[rgb]{0.00,0.50,0.00}{##1}}}
\expandafter\def\csname PY@tok@nf\endcsname{\def\PY@tc##1{\textcolor[rgb]{0.00,0.00,1.00}{##1}}}
\expandafter\def\csname PY@tok@nc\endcsname{\let\PY@bf=\textbf\def\PY@tc##1{\textcolor[rgb]{0.00,0.00,1.00}{##1}}}
\expandafter\def\csname PY@tok@nn\endcsname{\let\PY@bf=\textbf\def\PY@tc##1{\textcolor[rgb]{0.00,0.00,1.00}{##1}}}
\expandafter\def\csname PY@tok@ne\endcsname{\let\PY@bf=\textbf\def\PY@tc##1{\textcolor[rgb]{0.82,0.25,0.23}{##1}}}
\expandafter\def\csname PY@tok@nv\endcsname{\def\PY@tc##1{\textcolor[rgb]{0.10,0.09,0.49}{##1}}}
\expandafter\def\csname PY@tok@no\endcsname{\def\PY@tc##1{\textcolor[rgb]{0.53,0.00,0.00}{##1}}}
\expandafter\def\csname PY@tok@nl\endcsname{\def\PY@tc##1{\textcolor[rgb]{0.63,0.63,0.00}{##1}}}
\expandafter\def\csname PY@tok@ni\endcsname{\let\PY@bf=\textbf\def\PY@tc##1{\textcolor[rgb]{0.60,0.60,0.60}{##1}}}
\expandafter\def\csname PY@tok@na\endcsname{\def\PY@tc##1{\textcolor[rgb]{0.49,0.56,0.16}{##1}}}
\expandafter\def\csname PY@tok@nt\endcsname{\let\PY@bf=\textbf\def\PY@tc##1{\textcolor[rgb]{0.00,0.50,0.00}{##1}}}
\expandafter\def\csname PY@tok@nd\endcsname{\def\PY@tc##1{\textcolor[rgb]{0.67,0.13,1.00}{##1}}}
\expandafter\def\csname PY@tok@s\endcsname{\def\PY@tc##1{\textcolor[rgb]{0.73,0.13,0.13}{##1}}}
\expandafter\def\csname PY@tok@sd\endcsname{\let\PY@it=\textit\def\PY@tc##1{\textcolor[rgb]{0.73,0.13,0.13}{##1}}}
\expandafter\def\csname PY@tok@si\endcsname{\let\PY@bf=\textbf\def\PY@tc##1{\textcolor[rgb]{0.73,0.40,0.53}{##1}}}
\expandafter\def\csname PY@tok@se\endcsname{\let\PY@bf=\textbf\def\PY@tc##1{\textcolor[rgb]{0.73,0.40,0.13}{##1}}}
\expandafter\def\csname PY@tok@sr\endcsname{\def\PY@tc##1{\textcolor[rgb]{0.73,0.40,0.53}{##1}}}
\expandafter\def\csname PY@tok@ss\endcsname{\def\PY@tc##1{\textcolor[rgb]{0.10,0.09,0.49}{##1}}}
\expandafter\def\csname PY@tok@sx\endcsname{\def\PY@tc##1{\textcolor[rgb]{0.00,0.50,0.00}{##1}}}
\expandafter\def\csname PY@tok@m\endcsname{\def\PY@tc##1{\textcolor[rgb]{0.40,0.40,0.40}{##1}}}
\expandafter\def\csname PY@tok@gh\endcsname{\let\PY@bf=\textbf\def\PY@tc##1{\textcolor[rgb]{0.00,0.00,0.50}{##1}}}
\expandafter\def\csname PY@tok@gu\endcsname{\let\PY@bf=\textbf\def\PY@tc##1{\textcolor[rgb]{0.50,0.00,0.50}{##1}}}
\expandafter\def\csname PY@tok@gd\endcsname{\def\PY@tc##1{\textcolor[rgb]{0.63,0.00,0.00}{##1}}}
\expandafter\def\csname PY@tok@gi\endcsname{\def\PY@tc##1{\textcolor[rgb]{0.00,0.63,0.00}{##1}}}
\expandafter\def\csname PY@tok@gr\endcsname{\def\PY@tc##1{\textcolor[rgb]{1.00,0.00,0.00}{##1}}}
\expandafter\def\csname PY@tok@ge\endcsname{\let\PY@it=\textit}
\expandafter\def\csname PY@tok@gs\endcsname{\let\PY@bf=\textbf}
\expandafter\def\csname PY@tok@gp\endcsname{\let\PY@bf=\textbf\def\PY@tc##1{\textcolor[rgb]{0.00,0.00,0.50}{##1}}}
\expandafter\def\csname PY@tok@go\endcsname{\def\PY@tc##1{\textcolor[rgb]{0.53,0.53,0.53}{##1}}}
\expandafter\def\csname PY@tok@gt\endcsname{\def\PY@tc##1{\textcolor[rgb]{0.00,0.27,0.87}{##1}}}
\expandafter\def\csname PY@tok@err\endcsname{\def\PY@bc##1{\setlength{\fboxsep}{0pt}\fcolorbox[rgb]{1.00,0.00,0.00}{1,1,1}{\strut ##1}}}
\expandafter\def\csname PY@tok@kc\endcsname{\let\PY@bf=\textbf\def\PY@tc##1{\textcolor[rgb]{0.00,0.50,0.00}{##1}}}
\expandafter\def\csname PY@tok@kd\endcsname{\let\PY@bf=\textbf\def\PY@tc##1{\textcolor[rgb]{0.00,0.50,0.00}{##1}}}
\expandafter\def\csname PY@tok@kn\endcsname{\let\PY@bf=\textbf\def\PY@tc##1{\textcolor[rgb]{0.00,0.50,0.00}{##1}}}
\expandafter\def\csname PY@tok@kr\endcsname{\let\PY@bf=\textbf\def\PY@tc##1{\textcolor[rgb]{0.00,0.50,0.00}{##1}}}
\expandafter\def\csname PY@tok@bp\endcsname{\def\PY@tc##1{\textcolor[rgb]{0.00,0.50,0.00}{##1}}}
\expandafter\def\csname PY@tok@fm\endcsname{\def\PY@tc##1{\textcolor[rgb]{0.00,0.00,1.00}{##1}}}
\expandafter\def\csname PY@tok@vc\endcsname{\def\PY@tc##1{\textcolor[rgb]{0.10,0.09,0.49}{##1}}}
\expandafter\def\csname PY@tok@vg\endcsname{\def\PY@tc##1{\textcolor[rgb]{0.10,0.09,0.49}{##1}}}
\expandafter\def\csname PY@tok@vi\endcsname{\def\PY@tc##1{\textcolor[rgb]{0.10,0.09,0.49}{##1}}}
\expandafter\def\csname PY@tok@vm\endcsname{\def\PY@tc##1{\textcolor[rgb]{0.10,0.09,0.49}{##1}}}
\expandafter\def\csname PY@tok@sa\endcsname{\def\PY@tc##1{\textcolor[rgb]{0.73,0.13,0.13}{##1}}}
\expandafter\def\csname PY@tok@sb\endcsname{\def\PY@tc##1{\textcolor[rgb]{0.73,0.13,0.13}{##1}}}
\expandafter\def\csname PY@tok@sc\endcsname{\def\PY@tc##1{\textcolor[rgb]{0.73,0.13,0.13}{##1}}}
\expandafter\def\csname PY@tok@dl\endcsname{\def\PY@tc##1{\textcolor[rgb]{0.73,0.13,0.13}{##1}}}
\expandafter\def\csname PY@tok@s2\endcsname{\def\PY@tc##1{\textcolor[rgb]{0.73,0.13,0.13}{##1}}}
\expandafter\def\csname PY@tok@sh\endcsname{\def\PY@tc##1{\textcolor[rgb]{0.73,0.13,0.13}{##1}}}
\expandafter\def\csname PY@tok@s1\endcsname{\def\PY@tc##1{\textcolor[rgb]{0.73,0.13,0.13}{##1}}}
\expandafter\def\csname PY@tok@mb\endcsname{\def\PY@tc##1{\textcolor[rgb]{0.40,0.40,0.40}{##1}}}
\expandafter\def\csname PY@tok@mf\endcsname{\def\PY@tc##1{\textcolor[rgb]{0.40,0.40,0.40}{##1}}}
\expandafter\def\csname PY@tok@mh\endcsname{\def\PY@tc##1{\textcolor[rgb]{0.40,0.40,0.40}{##1}}}
\expandafter\def\csname PY@tok@mi\endcsname{\def\PY@tc##1{\textcolor[rgb]{0.40,0.40,0.40}{##1}}}
\expandafter\def\csname PY@tok@il\endcsname{\def\PY@tc##1{\textcolor[rgb]{0.40,0.40,0.40}{##1}}}
\expandafter\def\csname PY@tok@mo\endcsname{\def\PY@tc##1{\textcolor[rgb]{0.40,0.40,0.40}{##1}}}
\expandafter\def\csname PY@tok@ch\endcsname{\let\PY@it=\textit\def\PY@tc##1{\textcolor[rgb]{0.25,0.50,0.50}{##1}}}
\expandafter\def\csname PY@tok@cm\endcsname{\let\PY@it=\textit\def\PY@tc##1{\textcolor[rgb]{0.25,0.50,0.50}{##1}}}
\expandafter\def\csname PY@tok@cpf\endcsname{\let\PY@it=\textit\def\PY@tc##1{\textcolor[rgb]{0.25,0.50,0.50}{##1}}}
\expandafter\def\csname PY@tok@c1\endcsname{\let\PY@it=\textit\def\PY@tc##1{\textcolor[rgb]{0.25,0.50,0.50}{##1}}}
\expandafter\def\csname PY@tok@cs\endcsname{\let\PY@it=\textit\def\PY@tc##1{\textcolor[rgb]{0.25,0.50,0.50}{##1}}}

\def\PYZbs{\char`\\}
\def\PYZus{\char`\_}
\def\PYZob{\char`\{}
\def\PYZcb{\char`\}}
\def\PYZca{\char`\^}
\def\PYZam{\char`\&}
\def\PYZlt{\char`\<}
\def\PYZgt{\char`\>}
\def\PYZsh{\char`\#}
\def\PYZpc{\char`\%}
\def\PYZdl{\char`\$}
\def\PYZhy{\char`\-}
\def\PYZsq{\char`\'}
\def\PYZdq{\char`\"}
\def\PYZti{\char`\~}
% for compatibility with earlier versions
\def\PYZat{@}
\def\PYZlb{[}
\def\PYZrb{]}
\makeatother


    % Exact colors from NB
    \definecolor{incolor}{rgb}{0.0, 0.0, 0.5}
    \definecolor{outcolor}{rgb}{0.545, 0.0, 0.0}



    
    % Prevent overflowing lines due to hard-to-break entities
    \sloppy 
    % Setup hyperref package
    \hypersetup{
      breaklinks=true,  % so long urls are correctly broken across lines
      colorlinks=true,
      urlcolor=urlcolor,
      linkcolor=linkcolor,
      citecolor=citecolor,
      }
    % Slightly bigger margins than the latex defaults
    
    \geometry{verbose,tmargin=1in,bmargin=1in,lmargin=1in,rmargin=1in}
    
    

    \begin{document}
    
    
    \maketitle
    
    

    
     "Modeling Runners' Times in the Cherry Blossom Race, CaseStudy 8"

\textbf{Kevin Okiah, Shravan Kuchkula}

\textbf{06/30/2018}

    \textbf{NOTE}

We are answering Question 10

We have seen that the 1999 runners were typically older than the 2012
runners. Compare the age distribution of the runners across all 14 years
of the races. Use quantile-quantile plots, boxplots, and density curves
to make your comparisons. How do the distributions change over the
years? Was it a gradual change?"

    \subsection{Abstract}\label{abstract}

    In this case study, Cherry Blossom Ten Mile Run race results from 1999
to 2012 are used to study how the age distribution of the runners
changes over the years. The focus of the study is parsing data publicly
available in the Web and then getting that data into the correct format
to be used for analysis. To explore how age distributions change over
the years, several plots are used to compare them for all runners across
14 years. Close attention is paid first to formatting the data as what
information is recorded and how the data is formatted changes year after
year.

After successful data acquisition into data frames of race results in R,
results are analyzed and compared using density curves,
quantile-quantile plots, and boxplots. In this way, we are able to
visualize the data for tens of thousands of observations to the examine
age distribution.

    \subsection{Introduction}\label{introduction}

    In this era of `free and ubiquitous data,' there is tremendous potential
in seeking out data to bring insight to a problem we are working on
professionally or to a topic of personal interest. For example, we are
interested in understanding how the age distribution of runners changes
over the years. One source of data about this comes from road races.
Hundreds of thousands of people participate in road races each year; the
race organizers collect information about the runners' times and often
publish individual-level data on the Web. These freely accessible data
may provide us with insights to our question about age distribution.

One example of the many annual road races is the Cherry Blossom Ten Mile
Run held in Washington D.C. in early April when the cherry trees are
typically in bloom. The Cherry Blossom started in 1973 as a training run
for elite runners who were plan- ning to compete in the Boston Marathon.
It has since grown in popularity and in 2012 nearly 17,000 runners
ranging in age from 9 to 89 participated. After each year's race, the
organizers publish the results at http://www.cherryblossom.org/. These
data offer a tremendous resource for learning about how the age
distribution changes over the years.

    \subsection{Methods}\label{methods}

    The steps used for this analysis were: * data acquisition; * data
cleanup and outlier removal; * density curves analysis; *
quantile-quantile plots analysis; * boxplots analysis.

    \begin{itemize}
\tightlist
\item
  \textbf{Data acquisition}
\end{itemize}

    The methods provided in the Data Science in R text were utilized to load
runners from 1999 to 2012. We have excluded the scraping process from
this analysis as we leveraged the working code provided by Prof. Slater
and the data is loaded using an external \texttt{SourceAndCleanData.R}
to source men's data and \texttt{SourceAndCleanData\_Women.R} to source
women's data which allows us to focus on our analysis of age
distributions, what question 10 calls for.

    \begin{Verbatim}[commandchars=\\\{\}]
{\color{incolor}In [{\color{incolor}24}]:} \PY{c+c1}{\PYZsh{}\PYZsh{}\PYZsh{}\PYZsh{} load required packages}
         pacman\PY{o}{::}p\PYZus{}load\PY{p}{(}pacman\PY{p}{,}plotly\PY{p}{,}ggplot2\PY{p}{,} dplyr\PY{p}{,}formattable\PY{p}{)}\PY{c+c1}{\PYZsh{}, plyr)}
\end{Verbatim}

#Sourcing data from the web. We have commented this out 
source('SourceAndCleanData.R', echo = FALSE)
source('SourceAndCleanData_Women.R', echo = FALSE)
    \begin{Verbatim}[commandchars=\\\{\}]
{\color{incolor}In [{\color{incolor}25}]:} \PY{k+kp}{load}\PY{p}{(}\PY{l+s}{\PYZdq{}}\PY{l+s}{cbMen.rda\PYZdq{}}\PY{p}{)}
         \PY{k+kp}{load}\PY{p}{(}\PY{l+s}{\PYZdq{}}\PY{l+s}{cbWomen.rda\PYZdq{}}\PY{p}{)}
\end{Verbatim}


    After scraping the data for male and female runners from the website and
creating the dataframes, both the dataframes are merged to facilitate
analysis of age distribution for all the runners.

    \begin{Verbatim}[commandchars=\\\{\}]
{\color{incolor}In [{\color{incolor}26}]:} \PY{c+c1}{\PYZsh{}merge the two datasets}
         merged\PY{o}{\PYZlt{}\PYZhy{}}\PY{k+kp}{rbind}\PY{p}{(}cbMen\PY{p}{,} cbWomen\PY{p}{)}
\end{Verbatim}


    \begin{itemize}
\tightlist
\item
  ** Data Cleanup and Outlier Removal**
\end{itemize}

    Before proceeding with the analysis of age distribution, the dataset is
cleaned by converting the char types to factors; handling missing data
and removing outliers.

    \begin{Verbatim}[commandchars=\\\{\}]
{\color{incolor}In [{\color{incolor}27}]:} \PY{c+c1}{\PYZsh{}update the column types}
         merged\PY{o}{\PYZdl{}}sex\PY{o}{\PYZlt{}\PYZhy{}}\PY{k+kp}{as.factor}\PY{p}{(}merged\PY{o}{\PYZdl{}}sex\PY{p}{)}
         merged\PY{o}{\PYZdl{}}home\PY{o}{\PYZlt{}\PYZhy{}}\PY{k+kp}{as.factor}\PY{p}{(}merged\PY{o}{\PYZdl{}}home\PY{p}{)}
         merged\PY{o}{\PYZdl{}}name\PY{o}{\PYZlt{}\PYZhy{}}\PY{k+kp}{as.factor}\PY{p}{(}merged\PY{o}{\PYZdl{}}name\PY{p}{)}
         \PY{c+c1}{\PYZsh{}merged\PYZdl{}runTime\PYZlt{}\PYZhy{}as.numeric(merged\PYZdl{}runTime)}
\end{Verbatim}


    \begin{quote}
\textbf{\emph{Missing Data}}
\end{quote}

    \begin{Verbatim}[commandchars=\\\{\}]
{\color{incolor}In [{\color{incolor}28}]:} \PY{k+kp}{print}\PY{p}{(}\PY{l+s}{\PYZdq{}}\PY{l+s}{Summary Statistics\PYZdq{}}\PY{p}{)}
         \PY{k+kp}{summary}\PY{p}{(}merged\PY{p}{)}
         \PY{k+kp}{print}\PY{p}{(}\PY{l+s}{\PYZdq{}}\PY{l+s}{Missing Values Summary\PYZdq{}}\PY{p}{)}
         na\PYZus{}count \PY{o}{\PYZlt{}\PYZhy{}}\PY{k+kp}{lapply}\PY{p}{(}merged\PY{p}{,} \PY{k+kr}{function}\PY{p}{(}y\PY{p}{)} \PY{k+kp}{sum}\PY{p}{(}\PY{k+kp}{length}\PY{p}{(}\PY{k+kp}{which}\PY{p}{(}\PY{k+kp}{is.na}\PY{p}{(}y\PY{p}{)}\PY{p}{)}\PY{p}{)}\PY{p}{)}\PY{p}{)}
         na\PYZus{}count \PY{o}{\PYZlt{}\PYZhy{}} \PY{k+kt}{data.frame}\PY{p}{(}na\PYZus{}count\PY{p}{)}
         na\PYZus{}count
\end{Verbatim}


    \begin{Verbatim}[commandchars=\\\{\}]
[1] "Summary Statistics"

    \end{Verbatim}

    
    \begin{verbatim}
      year      sex                            name       
 Min.   :1999   F:70141   Michael Smith          :    17  
 1st Qu.:2004   M:70070   David Smith            :    14  
 Median :2008             Jennifer Johnson       :    14  
 Mean   :2007             John Kelly             :    14  
 3rd Qu.:2010             Michael Brown          :    14  
 Max.   :2012             Brian Murphy           :    12  
                          (Other)                :140126  
                    home             age           runTime      
 Washington DC        : 10732   Min.   : 0.00   Min.   :  1.00  
 Washington DC        : 10195   1st Qu.:28.00   1st Qu.: 80.30  
 Arlington VA         :  6403   Median :34.00   Median : 90.97  
 Arlington VA         :  5570   Mean   :36.23   Mean   : 89.22  
 Alexandria VA        :  3855   3rd Qu.:43.00   3rd Qu.:101.55  
 Alexandria VA        :  2587   Max.   :89.00   Max.   :177.52  
 (Other)              :100869   NA's   :44      NA's   :8400    
    \end{verbatim}

    
    \begin{Verbatim}[commandchars=\\\{\}]
[1] "Missing Values Summary"

    \end{Verbatim}

    \begin{tabular}{r|llllll}
 year & sex & name & home & age & runTime\\
\hline
	 0    & 0    & 0    & 0    & 44   & 8400\\
\end{tabular}


    
    A quick summary on the data gives us some insight into the data. We note
the existence of Missing observations in \texttt{age} and
\texttt{runtime}. 44 observations have missing age. These records are
not useful for visualizing age distributions, and thus removed from the
dataframe. Using \texttt{complete.cases} the observations which do not
have age have been removed.

    \begin{Verbatim}[commandchars=\\\{\}]
{\color{incolor}In [{\color{incolor}29}]:} \PY{k+kp}{print}\PY{p}{(}\PY{l+s}{\PYZdq{}}\PY{l+s}{Data  Missing Values\PYZdq{}}\PY{p}{)}
         \PY{k+kp}{print}\PY{p}{(}\PY{k+kp}{nrow}\PY{p}{(}merged\PY{p}{)}\PY{p}{)}
         merged\PY{o}{\PYZlt{}\PYZhy{}}merged\PY{p}{[}complete.cases\PY{p}{(}merged\PY{o}{\PYZdl{}}age\PY{p}{)}\PY{p}{,} \PY{p}{]}
         
         \PY{k+kp}{print}\PY{p}{(}\PY{l+s}{\PYZdq{}}\PY{l+s}{Clean data: \PYZdq{}}\PY{p}{)}
         \PY{k+kp}{print}\PY{p}{(}\PY{k+kp}{nrow}\PY{p}{(}merged\PY{p}{)}\PY{p}{)}
\end{Verbatim}


    \begin{Verbatim}[commandchars=\\\{\}]
[1] "Data  Missing Values"
[1] 140211
[1] "Clean data: "
[1] 140167

    \end{Verbatim}

    \begin{Verbatim}[commandchars=\\\{\}]
{\color{incolor}In [{\color{incolor}30}]:} \PY{k+kp}{head}\PY{p}{(}merged\PY{p}{)}
         \PY{k+kp}{tail}\PY{p}{(}merged\PY{p}{)}
         \PY{k+kp}{print}\PY{p}{(}\PY{l+s}{\PYZdq{}}\PY{l+s}{Missing Values Summary\PYZdq{}}\PY{p}{)}
         na\PYZus{}count \PY{o}{\PYZlt{}\PYZhy{}}\PY{k+kp}{lapply}\PY{p}{(}merged\PY{p}{,} \PY{k+kr}{function}\PY{p}{(}y\PY{p}{)} \PY{k+kp}{sum}\PY{p}{(}\PY{k+kp}{length}\PY{p}{(}\PY{k+kp}{which}\PY{p}{(}\PY{k+kp}{is.na}\PY{p}{(}y\PY{p}{)}\PY{p}{)}\PY{p}{)}\PY{p}{)}\PY{p}{)}
         na\PYZus{}count \PY{o}{\PYZlt{}\PYZhy{}} \PY{k+kt}{data.frame}\PY{p}{(}na\PYZus{}count\PY{p}{)}
         na\PYZus{}count
         \PY{k+kp}{summary}\PY{p}{(}merged\PY{p}{)}
\end{Verbatim}


    \begin{tabular}{r|llllll}
  & year & sex & name & home & age & runTime\\
\hline
	1999.1 & 1999                   & M                      & Worku Bikila           & Ethiopia               & 28                     & 46.98333              \\
	1999.2 & 1999                   & M                      & Lazarus Nyakeraka      & Kenya                  & 24                     & 47.01667              \\
	1999.3 & 1999                   & M                      & James Kariuki          & Kenya                  & 27                     & 47.05000              \\
	1999.4 & 1999                   & M                      & William Kiptum         & Kenya                  & 28                     & 47.11667              \\
	1999.5 & 1999                   & M                      & Joseph Kimani          & Kenya                  & 26                     & 47.51667              \\
	1999.6 & 1999                   & M                      & Josphat Machuka        & Kenya                  & 25                     & 47.55000              \\
\end{tabular}


    
    \begin{tabular}{r|llllll}
  & year & sex & name & home & age & runTime\\
\hline
	2012.9025 & 2012                    & F                       & Monica Carmean          & Washington DC           & 25                      & 144.7500               \\
	2012.9026 & 2012                    & F                       & Erin Grandstaff         & Chevy Chase MD          & 28                      & 144.7500               \\
	2012.9027 & 2012                    & F                       & Catherine Hanmer        & Astoria NY              & 36                      & 146.7500               \\
	2012.9028 & 2012                    & F                       & Kat Silvia              & Lisle IL                & 27                      & 146.8500               \\
	2012.9029 & 2012                    & F                       & Tram Anh Tran           & Fairfax VA              & 30                      & 150.5667               \\
	2012.9030 & 2012                    & F                       & Margie Rodan            & Potomac MD              & 53                      & 152.2667               \\
\end{tabular}


    
    \begin{Verbatim}[commandchars=\\\{\}]
[1] "Missing Values Summary"

    \end{Verbatim}

    \begin{tabular}{r|llllll}
 year & sex & name & home & age & runTime\\
\hline
	 0    & 0    & 0    & 0    & 0    & 8399\\
\end{tabular}


    
    
    \begin{verbatim}
      year      sex                            name       
 Min.   :1999   F:70120   Michael Smith          :    17  
 1st Qu.:2004   M:70047   David Smith            :    14  
 Median :2008             Jennifer Johnson       :    14  
 Mean   :2007             John Kelly             :    14  
 3rd Qu.:2010             Michael Brown          :    14  
 Max.   :2012             Brian Murphy           :    12  
                          (Other)                :140082  
                    home             age           runTime      
 Washington DC        : 10732   Min.   : 0.00   Min.   :  1.00  
 Washington DC        : 10186   1st Qu.:28.00   1st Qu.: 80.30  
 Arlington VA         :  6400   Median :34.00   Median : 90.97  
 Arlington VA         :  5569   Mean   :36.23   Mean   : 89.22  
 Alexandria VA        :  3854   3rd Qu.:43.00   3rd Qu.:101.55  
 Alexandria VA        :  2587   Max.   :89.00   Max.   :177.52  
 (Other)              :100839                   NA's   :8399    
    \end{verbatim}

    
    Our resulting dataset with complete cases has 140167 observations.

    From the summary statistics above, we also notice a pretty small runtime
of 1 which is likely either due to a non-starter or a data entry error.
Digging into the runtime variable, we have 4312 obsevations with a
runtime of \textless{}3. Taking a closer look , all this happens for
year 2006 women data except for one from the Mens data. Since the focus
of our analysis is change in age of the participants over the duration
of the competition 1999-2012, we decide not to do cleaning or further
analysis on the runTime parameter.

    \begin{Verbatim}[commandchars=\\\{\}]
{\color{incolor}In [{\color{incolor}31}]:} \PY{c+c1}{\PYZsh{} showing runtime values less than 30}
         \PY{k+kp}{head}\PY{p}{(}merged\PY{p}{[}\PY{k+kp}{which}\PY{p}{(}merged\PY{o}{\PYZdl{}}runTime\PY{o}{\PYZlt{}}\PY{l+m}{30}\PY{p}{)}\PY{p}{,}\PY{p}{]}\PY{p}{)}
         \PY{k+kp}{nrow}\PY{p}{(}merged\PY{p}{[}\PY{k+kp}{which}\PY{p}{(}merged\PY{o}{\PYZdl{}}runTime\PY{o}{\PYZlt{}}\PY{l+m}{30}\PY{p}{)}\PY{p}{,}\PY{p}{]}\PY{p}{)}
\end{Verbatim}


    \begin{tabular}{r|llllll}
  & year & sex & name & home & age & runTime\\
\hline
	2001.2250 & 2001                    & M                       & Peter HUI               & Silver Spring MD        & 70                      & 1.5                    \\
	2006.15100 & 2006                    & F                       & Vanessa Hunter          & Arlington VA            & 29                      & 1.0                    \\
	2006.16100 & 2006                    & F                       & Laura Turner            & Washington DC           & 24                      & 1.0                    \\
	2006.17100 & 2006                    & F                       & Lisa Thomas             & Alexandria VA           & 29                      & 1.0                    \\
	2006.18100 & 2006                    & F                       & Diana Pool              & Westminster MD          & 24                      & 1.0                    \\
	2006.19100 & 2006                    & F                       & Ann Reed                & Columbia MD             & 31                      & 1.0                    \\
\end{tabular}


    
    4312

    
    \begin{quote}
\textbf{Outliers}
\end{quote}

    Looking at the age variable we note that few observations are below the
age of 10. Although there could be a young child in the run, there are
several of these values with an age between 0 and 4 which looks pretty
odd. For the sake of narrowing our analysis to the general population,
We decide to set our lower age limit at 10 years old to describe general
population.

    \begin{Verbatim}[commandchars=\\\{\}]
{\color{incolor}In [{\color{incolor}32}]:} par\PY{p}{(}mfrow\PY{o}{=}\PY{k+kt}{c}\PY{p}{(}\PY{l+m}{2}\PY{p}{,}\PY{l+m}{2}\PY{p}{)}\PY{p}{)}
         \PY{k+kp}{options}\PY{p}{(}repr.plot.width\PY{o}{=}\PY{l+m}{10}\PY{p}{,} repr.plot.height\PY{o}{=}\PY{l+m}{8}\PY{p}{)}
         boxplot\PY{p}{(}age\PY{o}{\PYZti{}}year\PY{p}{,}data\PY{o}{=}merged\PY{p}{[}\PY{k+kp}{which}\PY{p}{(}merged\PY{o}{\PYZdl{}}sex\PY{o}{==}\PY{l+s}{\PYZdq{}}\PY{l+s}{M\PYZdq{}}\PY{p}{)}\PY{p}{,}\PY{p}{]}\PY{p}{,} main\PY{o}{=}\PY{l+s}{\PYZdq{}}\PY{l+s}{MEN: Cherry Blossom Age of runners by year with outliers\PYZdq{}}\PY{p}{,} xlab\PY{o}{=}\PY{l+s}{\PYZdq{}}\PY{l+s}{Year\PYZdq{}}\PY{p}{,} ylab\PY{o}{=}\PY{l+s}{\PYZdq{}}\PY{l+s}{Age\PYZdq{}}\PY{p}{)}
         boxplot\PY{p}{(}age\PY{o}{\PYZti{}}year\PY{p}{,}data\PY{o}{=}merged\PY{p}{[}\PY{k+kp}{which}\PY{p}{(}merged\PY{o}{\PYZdl{}}sex\PY{o}{==}\PY{l+s}{\PYZdq{}}\PY{l+s}{F\PYZdq{}}\PY{p}{)}\PY{p}{,}\PY{p}{]}\PY{p}{,} main\PY{o}{=}\PY{l+s}{\PYZdq{}}\PY{l+s}{WOMEN: Cherry Blossom Age of runners by year with outliers\PYZdq{}}\PY{p}{,} xlab\PY{o}{=}\PY{l+s}{\PYZdq{}}\PY{l+s}{Year\PYZdq{}}\PY{p}{,} ylab\PY{o}{=}\PY{l+s}{\PYZdq{}}\PY{l+s}{Age\PYZdq{}}\PY{p}{)}
         
         merged\PY{o}{\PYZlt{}\PYZhy{}}merged\PY{p}{[}\PY{k+kp}{which}\PY{p}{(}merged\PY{o}{\PYZdl{}}age \PY{o}{\PYZgt{}} \PY{l+m}{10}\PY{p}{)}\PY{p}{,} \PY{p}{]}
         boxplot\PY{p}{(}age\PY{o}{\PYZti{}}year\PY{p}{,}data\PY{o}{=}merged\PY{p}{[}\PY{k+kp}{which}\PY{p}{(}merged\PY{o}{\PYZdl{}}sex\PY{o}{==}\PY{l+s}{\PYZdq{}}\PY{l+s}{M\PYZdq{}}\PY{p}{)}\PY{p}{,}\PY{p}{]}\PY{p}{,} main\PY{o}{=}\PY{l+s}{\PYZdq{}}\PY{l+s}{MEN:Cherry Blossom Age of runners by year Clean\PYZdq{}}\PY{p}{,} xlab\PY{o}{=}\PY{l+s}{\PYZdq{}}\PY{l+s}{Year\PYZdq{}}\PY{p}{,} ylab\PY{o}{=}\PY{l+s}{\PYZdq{}}\PY{l+s}{Age\PYZdq{}}\PY{p}{)}
         \PY{c+c1}{\PYZsh{}merged\PYZlt{}\PYZhy{}merged[which(merged\PYZdl{}age \PYZgt{} 10), ]}
         boxplot\PY{p}{(}age\PY{o}{\PYZti{}}year\PY{p}{,}data\PY{o}{=}merged\PY{p}{[}\PY{k+kp}{which}\PY{p}{(}merged\PY{o}{\PYZdl{}}sex\PY{o}{==}\PY{l+s}{\PYZdq{}}\PY{l+s}{F\PYZdq{}}\PY{p}{)}\PY{p}{,}\PY{p}{]}\PY{p}{,} main\PY{o}{=}\PY{l+s}{\PYZdq{}}\PY{l+s}{WOMEN:Cherry Blossom Age of runners by year Clean\PYZdq{}}\PY{p}{,} xlab\PY{o}{=}\PY{l+s}{\PYZdq{}}\PY{l+s}{Year\PYZdq{}}\PY{p}{,} ylab\PY{o}{=}\PY{l+s}{\PYZdq{}}\PY{l+s}{Age\PYZdq{}}\PY{p}{)}
         
         \PY{k+kp}{print}\PY{p}{(}\PY{k+kp}{nrow}\PY{p}{(}merged\PY{p}{)}\PY{p}{)}
\end{Verbatim}


    \begin{Verbatim}[commandchars=\\\{\}]
[1] 140152

    \end{Verbatim}

    \begin{center}
    \adjustimage{max size={0.9\linewidth}{0.9\paperheight}}{output_28_1.png}
    \end{center}
    { \hspace*{\fill} \\}
    
    Having removed the few outliers, our final dataset has 140152
observations. Next we move to our analysis of age distributions for each
year.

    \begin{itemize}
\tightlist
\item
  \textbf{density curves analysis}
\end{itemize}

    \begin{Verbatim}[commandchars=\\\{\}]
{\color{incolor}In [{\color{incolor}33}]:} \PY{k+kp}{summary}\PY{p}{(}merged\PY{o}{\PYZdl{}}age\PY{p}{)}
\end{Verbatim}


    
    \begin{verbatim}
   Min. 1st Qu.  Median    Mean 3rd Qu.    Max. 
  11.00   28.00   34.00   36.23   43.00   89.00 
    \end{verbatim}

    
    Beginining with simple statistcs, we note mean of 36.23 and a median of
34 for age. The mean being larger than a median indicates a potential
right skewed distribution. Below we proceed to group age into bins of
decades (10-20, 20-30, 30-40, 40-50, 50-60, 60-70, 70-80 and 80-90) for
easy vizualization of how age group varies. Visualizing the binned ages,
we are able to confirm a conclusion from the mean and median comparisons
where the age distribution has a larger tail extending from the 30 to 90
compared to that from 30 to 10 for both genders. Majority of the
participants fall in the 30-40 age group for men and 20-30 for women.

    \begin{Verbatim}[commandchars=\\\{\}]
{\color{incolor}In [{\color{incolor}34}]:} merged\PY{o}{\PYZdl{}}ageCat \PY{o}{=} \PY{k+kp}{cut}\PY{p}{(}merged\PY{o}{\PYZdl{}}age\PY{p}{,} breaks \PY{o}{=} \PY{k+kt}{c}\PY{p}{(}\PY{k+kp}{seq}\PY{p}{(}\PY{l+m}{10}\PY{p}{,} \PY{l+m}{80}\PY{p}{,} \PY{l+m}{10}\PY{p}{)}\PY{p}{,} \PY{l+m}{90}\PY{p}{)}\PY{p}{)}
         \PY{k+kp}{head}\PY{p}{(}merged\PY{p}{)}
\end{Verbatim}


    \begin{tabular}{r|lllllll}
  & year & sex & name & home & age & runTime & ageCat\\
\hline
	1999.1 & 1999                   & M                      & Worku Bikila           & Ethiopia               & 28                     & 46.98333               & (20,30{]}             \\
	1999.2 & 1999                   & M                      & Lazarus Nyakeraka      & Kenya                  & 24                     & 47.01667               & (20,30{]}             \\
	1999.3 & 1999                   & M                      & James Kariuki          & Kenya                  & 27                     & 47.05000               & (20,30{]}             \\
	1999.4 & 1999                   & M                      & William Kiptum         & Kenya                  & 28                     & 47.11667               & (20,30{]}             \\
	1999.5 & 1999                   & M                      & Joseph Kimani          & Kenya                  & 26                     & 47.51667               & (20,30{]}             \\
	1999.6 & 1999                   & M                      & Josphat Machuka        & Kenya                  & 25                     & 47.55000               & (20,30{]}             \\
\end{tabular}


    
    \begin{Verbatim}[commandchars=\\\{\}]
{\color{incolor}In [{\color{incolor}35}]:} ageCat \PY{o}{=} \PY{k+kp}{cut}\PY{p}{(}merged\PY{p}{[}\PY{k+kp}{which}\PY{p}{(}merged\PY{o}{\PYZdl{}}sex\PY{o}{==}\PY{l+s}{\PYZsq{}}\PY{l+s}{M\PYZsq{}}\PY{p}{)}\PY{p}{,}\PY{p}{]}\PY{o}{\PYZdl{}}age\PY{p}{,} breaks \PY{o}{=} \PY{k+kt}{c}\PY{p}{(}\PY{k+kp}{seq}\PY{p}{(}\PY{l+m}{10}\PY{p}{,} \PY{l+m}{80}\PY{p}{,} \PY{l+m}{10}\PY{p}{)}\PY{p}{,} \PY{l+m}{90}\PY{p}{)}\PY{p}{)}
         
         bins\PY{o}{=}\PY{k+kp}{as.data.frame}\PY{p}{(}\PY{k+kp}{table}\PY{p}{(}ageCat\PY{p}{)}\PY{p}{)}
         
         bins.p \PY{o}{\PYZlt{}\PYZhy{}}ggplot\PY{p}{(}bins\PY{p}{,} aes\PY{p}{(}x\PY{o}{=}ageCat\PY{p}{,} y\PY{o}{=}Freq\PY{p}{)}\PY{p}{)} \PY{o}{+} geom\PYZus{}bar\PY{p}{(}stat\PY{o}{=}\PY{l+s}{\PYZdq{}}\PY{l+s}{identity\PYZdq{}}\PY{p}{)}\PY{o}{+}\PY{c+c1}{\PYZsh{}facet\PYZus{}wrap(\PYZti{}as.factor(merged\PYZdl{}sex), nrow=2)}
                 ggtitle\PY{p}{(}\PY{l+s}{\PYZdq{}}\PY{l+s}{Men\PYZsq{}s Frequency of Age groups\PYZdq{}}\PY{p}{)}
         \PY{k+kp}{suppressMessages}\PY{p}{(}bins.p\PY{o}{\PYZlt{}\PYZhy{}}ggplotly\PY{p}{(}bins.p\PY{p}{)}\PY{p}{)}
         bins.p
         
         ageCat \PY{o}{=} \PY{k+kp}{cut}\PY{p}{(}merged\PY{p}{[}\PY{k+kp}{which}\PY{p}{(}merged\PY{o}{\PYZdl{}}sex\PY{o}{==}\PY{l+s}{\PYZsq{}}\PY{l+s}{F\PYZsq{}}\PY{p}{)}\PY{p}{,}\PY{p}{]}\PY{o}{\PYZdl{}}age\PY{p}{,} breaks \PY{o}{=} \PY{k+kt}{c}\PY{p}{(}\PY{k+kp}{seq}\PY{p}{(}\PY{l+m}{10}\PY{p}{,} \PY{l+m}{80}\PY{p}{,} \PY{l+m}{10}\PY{p}{)}\PY{p}{,} \PY{l+m}{90}\PY{p}{)}\PY{p}{)}
         
         bins\PY{o}{=}\PY{k+kp}{as.data.frame}\PY{p}{(}\PY{k+kp}{table}\PY{p}{(}ageCat\PY{p}{)}\PY{p}{)}
         
         bins.p \PY{o}{\PYZlt{}\PYZhy{}}ggplot\PY{p}{(}bins\PY{p}{,} aes\PY{p}{(}x\PY{o}{=}ageCat\PY{p}{,} y\PY{o}{=}Freq\PY{p}{)}\PY{p}{)} \PY{o}{+} geom\PYZus{}bar\PY{p}{(}stat\PY{o}{=}\PY{l+s}{\PYZdq{}}\PY{l+s}{identity\PYZdq{}}\PY{p}{)}\PY{o}{+}\PY{c+c1}{\PYZsh{}facet\PYZus{}wrap(\PYZti{}as.factor(merged\PYZdl{}sex), nrow=2)}
                 ggtitle\PY{p}{(}\PY{l+s}{\PYZdq{}}\PY{l+s}{Women\PYZsq{}s Frequency of Age groups\PYZdq{}}\PY{p}{)}
         \PY{k+kp}{suppressMessages}\PY{p}{(}bins.p\PY{o}{\PYZlt{}\PYZhy{}}ggplotly\PY{p}{(}bins.p\PY{p}{)}\PY{p}{)}
         bins.p
\end{Verbatim}


    
    \begin{verbatim}
HTML widgets cannot be represented in plain text (need html)
    \end{verbatim}

    
    
    \begin{verbatim}
HTML widgets cannot be represented in plain text (need html)
    \end{verbatim}

    
    We are also interested in how the age demographics of runners changes
with years. To achieve that we used stacked density plots shown below
using ggplot and plotly libraries. Each colored density is an age
distribution for a given year color coded as per the key.

On the density curves below, we observe the curves shift positions
gradually from right to left and density skewness elongates as we
progress from 1999 to 2012. This shift in density plot indicates that
there is a general decrease in mean age of particants in 1999 compared
to that of later years. This gradual decrease in age applies to both Men
and Women.

Although the mean distribution is shifting left, it is apparent that
participation amongst the age groups above 40 have not significantly
decreased. Because of this, we begin seeing the right-skew distribution
increase in the most recent years of the race.

Since we used plotly below, we can interactively explore the data and
even select a specific year to narrow you research to by clicking the
year on the key.

    \begin{Verbatim}[commandchars=\\\{\}]
{\color{incolor}In [{\color{incolor}53}]:} \PY{k+kp}{options}\PY{p}{(}repr.plot.width\PY{o}{=}\PY{l+m}{10}\PY{p}{,} repr.plot.height\PY{o}{=}\PY{l+m}{8}\PY{p}{)}
         merged\PY{o}{\PYZdl{}}year \PY{o}{\PYZlt{}\PYZhy{}} \PY{k+kp}{as.character}\PY{p}{(}merged\PY{o}{\PYZdl{}}year\PY{p}{)}
         age.d \PY{o}{=} ggplot\PY{p}{(}merged\PY{p}{,} aes\PY{p}{(}age\PY{p}{,} fill \PY{o}{=} \PY{k+kp}{factor}\PY{p}{(}year\PY{p}{)}\PY{p}{)}\PY{p}{)} \PY{o}{+} geom\PYZus{}density\PY{p}{(}col\PY{o}{=}\PY{k+kc}{NA}\PY{p}{,} alpha\PY{o}{=}\PY{l+m}{0.15}\PY{p}{)} \PY{o}{+} theme\PYZus{}light\PY{p}{(}\PY{p}{)} \PY{o}{+} 
                 scale\PYZus{}x\PYZus{}continuous\PY{p}{(}breaks \PY{o}{=} \PY{k+kp}{pretty}\PY{p}{(}merged\PY{o}{\PYZdl{}}age\PY{p}{,} n \PY{o}{=} \PY{l+m}{10}\PY{p}{)}\PY{p}{)}\PY{o}{+}facet\PYZus{}wrap\PY{p}{(}\PY{o}{\PYZti{}}\PY{k+kp}{as.factor}\PY{p}{(}sex\PY{p}{)}\PY{p}{,} nrow\PY{o}{=}\PY{l+m}{2}\PY{p}{)}\PY{o}{+}
                 ggtitle\PY{p}{(}\PY{l+s}{\PYZdq{}}\PY{l+s}{Density plot of Age by Year\PYZdq{}}\PY{p}{)}
         age.d \PY{o}{=} \PY{k+kp}{suppressMessages}\PY{p}{(}ggplotly\PY{p}{(}age.d\PY{p}{)}\PY{p}{)}
         age.d
\end{Verbatim}


    
    \begin{verbatim}
HTML widgets cannot be represented in plain text (need html)
    \end{verbatim}

    
    Across the board,Men participating in the race are much older than the
women in the race as depicted by the means of the density plots.

We verify our observations from visual inspection of density plots which
indicates that the core demographic for participating runners is
decreasing year after year by generating mean summary plot below.

    \begin{Verbatim}[commandchars=\\\{\}]
{\color{incolor}In [{\color{incolor}40}]:} temp\PY{o}{\PYZlt{}\PYZhy{}}merged\PY{o}{\PYZpc{}\PYZgt{}\PYZpc{}}
         select\PY{p}{(}age\PY{p}{,}sex\PY{p}{,} year\PY{p}{)}\PY{o}{\PYZpc{}\PYZgt{}\PYZpc{}}
         group\PYZus{}by\PY{p}{(}sex\PY{p}{,}year\PY{p}{)}\PY{o}{\PYZpc{}\PYZgt{}\PYZpc{}}
         summarise\PY{p}{(}AvgAge \PY{o}{=}\PY{k+kp}{mean}\PY{p}{(}age\PY{p}{)}\PY{p}{)}
\end{Verbatim}


    \begin{Verbatim}[commandchars=\\\{\}]
{\color{incolor}In [{\color{incolor}41}]:} \PY{k+kp}{options}\PY{p}{(}repr.plot.width\PY{o}{=}\PY{l+m}{10}\PY{p}{,} repr.plot.height\PY{o}{=}\PY{l+m}{8}\PY{p}{)}
         age.l\PY{o}{=}ggplot\PY{p}{(}data\PY{o}{=}temp\PY{p}{,} aes\PY{p}{(}x\PY{o}{=}year\PY{p}{,} y\PY{o}{=}AvgAge\PY{p}{,} group\PY{o}{=}sex\PY{p}{,} shape\PY{o}{=}sex\PY{p}{,} color\PY{o}{=}sex\PY{p}{)}\PY{p}{)} \PY{o}{+}
             geom\PYZus{}line\PY{p}{(}\PY{p}{)} \PY{o}{+}
             geom\PYZus{}point\PY{p}{(}\PY{p}{)}\PY{o}{+}
             ggtitle\PY{p}{(}\PY{l+s}{\PYZdq{}}\PY{l+s}{Mean Age line plot. Men vs Women\PYZdq{}}\PY{p}{)}
         
         age.l \PY{o}{=} \PY{k+kp}{suppressMessages}\PY{p}{(}ggplotly\PY{p}{(}age.l\PY{p}{)}\PY{p}{)}
         age.l
\end{Verbatim}


    
    \begin{verbatim}
HTML widgets cannot be represented in plain text (need html)
    \end{verbatim}

    
    The overall mean age line plot, confirms the same observation, that
there is a gradual decrease in the ages of the runners with the lowest
being 2010.

    \begin{Verbatim}[commandchars=\\\{\}]
{\color{incolor}In [{\color{incolor}54}]:} temp1\PY{o}{\PYZlt{}\PYZhy{}}merged\PY{o}{\PYZpc{}\PYZgt{}\PYZpc{}}
         select\PY{p}{(}age\PY{p}{,} year\PY{p}{)}\PY{o}{\PYZpc{}\PYZgt{}\PYZpc{}}
         group\PYZus{}by\PY{p}{(}year\PY{p}{)}\PY{o}{\PYZpc{}\PYZgt{}\PYZpc{}}
         summarise\PY{p}{(}AvgAge \PY{o}{=}\PY{k+kp}{mean}\PY{p}{(}age\PY{p}{)}\PY{p}{)}
\end{Verbatim}


    \begin{Verbatim}[commandchars=\\\{\}]
{\color{incolor}In [{\color{incolor}65}]:} \PY{k+kp}{options}\PY{p}{(}repr.plot.width\PY{o}{=}\PY{l+m}{10}\PY{p}{,} repr.plot.height\PY{o}{=}\PY{l+m}{8}\PY{p}{)}
         age.2\PY{o}{=}ggplot\PY{p}{(}data\PY{o}{=}temp1\PY{p}{,} aes\PY{p}{(}x\PY{o}{=}year\PY{p}{,} y\PY{o}{=}AvgAge\PY{p}{,} group\PY{o}{=}\PY{l+m}{1}\PY{p}{)}\PY{p}{)} \PY{o}{+}
             geom\PYZus{}line\PY{p}{(}color\PY{o}{=}\PY{l+s}{\PYZdq{}}\PY{l+s}{red\PYZdq{}}\PY{p}{)} \PY{o}{+}
             geom\PYZus{}point\PY{p}{(}color\PY{o}{=}\PY{l+s}{\PYZdq{}}\PY{l+s}{blue\PYZdq{}}\PY{p}{)} \PY{o}{+}
             ggtitle\PY{p}{(}\PY{l+s}{\PYZdq{}}\PY{l+s}{Mean Age line plot for both genders\PYZdq{}}\PY{p}{)}
         
         age.2 \PY{o}{=} \PY{k+kp}{suppressMessages}\PY{p}{(}ggplotly\PY{p}{(}age.2\PY{p}{)}\PY{p}{)}
         age.2
\end{Verbatim}


    
    \begin{verbatim}
HTML widgets cannot be represented in plain text (need html)
    \end{verbatim}

    
    \begin{itemize}
\tightlist
\item
  \textbf{quantile-quantile plots analysis}
\end{itemize}

A quantile-quantile plot can be used to assess if the age distributions
come from some theoritical distribution such as Normal distribution or
exponential. This is a quick way to compare the age distributions across
all the years and will help one answer if all the years had a similar
age distribution or not. For instance, looking at the
\texttt{QQ\ plot\ of\ Age\ by\ Year\ -\ Male}, one can see that from the
year 2006 to 2012 the points below the 25th quartile do not seem to form
a straight line. This indicates that more number of runners' age was
below the 25th quartile.

    \begin{Verbatim}[commandchars=\\\{\}]
{\color{incolor}In [{\color{incolor}45}]:} pacman\PY{o}{::}p\PYZus{}load\PY{p}{(}plyr\PY{p}{)}
         \PY{k+kp}{options}\PY{p}{(}repr.plot.width\PY{o}{=}\PY{l+m}{10}\PY{p}{,} repr.plot.height\PY{o}{=}\PY{l+m}{8}\PY{p}{)}
         cbMenQQ \PY{o}{\PYZlt{}\PYZhy{}} ddply\PY{p}{(}\PY{l+m}{.}data \PY{o}{=} merged\PY{p}{[}\PY{k+kp}{which}\PY{p}{(}merged\PY{o}{\PYZdl{}}sex\PY{o}{==}\PY{l+s}{\PYZsq{}}\PY{l+s}{M\PYZsq{}}\PY{p}{)}\PY{p}{,}\PY{p}{]}\PY{p}{,} \PY{l+m}{.}variables \PY{o}{=} \PY{l+m}{.}\PY{p}{(}year\PY{p}{)}\PY{p}{,}
                               \PY{k+kr}{function}\PY{p}{(}dsub\PY{p}{)}\PY{p}{\PYZob{}}
                                   q \PY{o}{\PYZlt{}\PYZhy{}} qqnorm\PY{p}{(}dsub\PY{o}{\PYZdl{}}age\PY{p}{,} plot \PY{o}{=} \PY{k+kc}{FALSE}\PY{p}{)}
                                   dsub\PY{o}{\PYZdl{}}xq \PY{o}{\PYZlt{}\PYZhy{}} \PY{k+kp}{q}\PY{o}{\PYZdl{}}x
                                   dsub
                               \PY{p}{\PYZcb{}}\PY{p}{)}
         
         age.qq \PY{o}{=} ggplot\PY{p}{(}data \PY{o}{=} cbMenQQ\PY{p}{,} aes\PY{p}{(}x \PY{o}{=} xq\PY{p}{,} y \PY{o}{=} age\PY{p}{,} color \PY{o}{=} year\PY{p}{)}\PY{p}{)} \PY{o}{+}
                         geom\PYZus{}point\PY{p}{(}alpha\PY{o}{=}\PY{l+m}{0.25}\PY{p}{)} \PY{o}{+}
                         geom\PYZus{}smooth\PY{p}{(}method \PY{o}{=} \PY{l+s}{\PYZdq{}}\PY{l+s}{lm\PYZdq{}}\PY{p}{,} se \PY{o}{=} \PY{k+kc}{FALSE}\PY{p}{)} \PY{o}{+}facet\PYZus{}wrap\PY{p}{(}\PY{o}{\PYZti{}}\PY{k+kp}{as.factor}\PY{p}{(}year\PY{p}{)}\PY{p}{,} nrow\PY{o}{=}\PY{l+m}{4}\PY{p}{)}\PY{o}{+}
                         xlab\PY{p}{(}\PY{l+s}{\PYZdq{}}\PY{l+s}{Normal Theoretical Quantiles\PYZdq{}}\PY{p}{)} \PY{o}{+}
                         ylab\PY{p}{(}\PY{l+s}{\PYZdq{}}\PY{l+s}{Normal Data Quantiles\PYZdq{}}\PY{p}{)}\PY{o}{+}
                         ggtitle\PY{p}{(}\PY{l+s}{\PYZdq{}}\PY{l+s}{QQ plot of Age by Year \PYZhy{} Male\PYZdq{}}\PY{p}{)}
         age.qq
         \PY{k+kn}{detach}\PY{p}{(}package\PY{o}{:}plyr\PY{p}{)}
\end{Verbatim}


    
    
    \begin{center}
    \adjustimage{max size={0.9\linewidth}{0.9\paperheight}}{output_44_1.png}
    \end{center}
    { \hspace*{\fill} \\}
    
    Similarly, QQ-plots have been compared for women's data over the years.
Interesting observersation here is that unlike the men's QQ-plots, the
women runner's ages tend to follow similar distributions except for 1999
and 2000.

    \begin{Verbatim}[commandchars=\\\{\}]
{\color{incolor}In [{\color{incolor}46}]:} pacman\PY{o}{::}p\PYZus{}load\PY{p}{(}plyr\PY{p}{)}
         \PY{k+kp}{options}\PY{p}{(}repr.plot.width\PY{o}{=}\PY{l+m}{10}\PY{p}{,} repr.plot.height\PY{o}{=}\PY{l+m}{8}\PY{p}{)}
         cbMenQQ \PY{o}{\PYZlt{}\PYZhy{}} ddply\PY{p}{(}\PY{l+m}{.}data \PY{o}{=} merged\PY{p}{[}\PY{k+kp}{which}\PY{p}{(}merged\PY{o}{\PYZdl{}}sex\PY{o}{==}\PY{l+s}{\PYZsq{}}\PY{l+s}{F\PYZsq{}}\PY{p}{)}\PY{p}{,}\PY{p}{]}\PY{p}{,} \PY{l+m}{.}variables \PY{o}{=} \PY{l+m}{.}\PY{p}{(}year\PY{p}{)}\PY{p}{,}
                               \PY{k+kr}{function}\PY{p}{(}dsub\PY{p}{)}\PY{p}{\PYZob{}}
                                   q \PY{o}{\PYZlt{}\PYZhy{}} qqnorm\PY{p}{(}dsub\PY{o}{\PYZdl{}}age\PY{p}{,} plot \PY{o}{=} \PY{k+kc}{FALSE}\PY{p}{)}
                                   dsub\PY{o}{\PYZdl{}}xq \PY{o}{\PYZlt{}\PYZhy{}} \PY{k+kp}{q}\PY{o}{\PYZdl{}}x
                                   dsub
                               \PY{p}{\PYZcb{}}\PY{p}{)}
         
         age.qq \PY{o}{=} ggplot\PY{p}{(}data \PY{o}{=} cbMenQQ\PY{p}{,} aes\PY{p}{(}x \PY{o}{=} xq\PY{p}{,} y \PY{o}{=} age\PY{p}{,} color \PY{o}{=} year\PY{p}{)}\PY{p}{)} \PY{o}{+}
                         geom\PYZus{}point\PY{p}{(}alpha\PY{o}{=}\PY{l+m}{0.25}\PY{p}{)} \PY{o}{+}
                         geom\PYZus{}smooth\PY{p}{(}method \PY{o}{=} \PY{l+s}{\PYZdq{}}\PY{l+s}{lm\PYZdq{}}\PY{p}{,} se \PY{o}{=} \PY{k+kc}{FALSE}\PY{p}{)} \PY{o}{+}facet\PYZus{}wrap\PY{p}{(}\PY{o}{\PYZti{}}\PY{k+kp}{as.factor}\PY{p}{(}year\PY{p}{)}\PY{p}{,} nrow\PY{o}{=}\PY{l+m}{4}\PY{p}{)}\PY{o}{+}
                         xlab\PY{p}{(}\PY{l+s}{\PYZdq{}}\PY{l+s}{Normal Theoretical Quantiles\PYZdq{}}\PY{p}{)} \PY{o}{+}
                         ylab\PY{p}{(}\PY{l+s}{\PYZdq{}}\PY{l+s}{Normal Data Quantiles\PYZdq{}}\PY{p}{)}\PY{o}{+}
                         ggtitle\PY{p}{(}\PY{l+s}{\PYZdq{}}\PY{l+s}{QQ plot of Age by Year \PYZhy{} Female\PYZdq{}}\PY{p}{)}
         age.qq
         \PY{k+kn}{detach}\PY{p}{(}package\PY{o}{:}plyr\PY{p}{)}
\end{Verbatim}


    
    
    \begin{center}
    \adjustimage{max size={0.9\linewidth}{0.9\paperheight}}{output_46_1.png}
    \end{center}
    { \hspace*{\fill} \\}
    
    The QQ-plots for both men and women when combined and compared, show the
general trend in the distribution - which is, that since 2004 there has
been a gradual decrease in the ages of runners.

    \begin{Verbatim}[commandchars=\\\{\}]
{\color{incolor}In [{\color{incolor}50}]:} pacman\PY{o}{::}p\PYZus{}load\PY{p}{(}plyr\PY{p}{)}
         \PY{k+kp}{options}\PY{p}{(}repr.plot.width\PY{o}{=}\PY{l+m}{10}\PY{p}{,} repr.plot.height\PY{o}{=}\PY{l+m}{8}\PY{p}{)}
         cbMFQQ \PY{o}{\PYZlt{}\PYZhy{}} ddply\PY{p}{(}\PY{l+m}{.}data \PY{o}{=} merged\PY{p}{,} \PY{l+m}{.}variables \PY{o}{=} \PY{l+m}{.}\PY{p}{(}year\PY{p}{)}\PY{p}{,}
                               \PY{k+kr}{function}\PY{p}{(}dsub\PY{p}{)}\PY{p}{\PYZob{}}
                                   q \PY{o}{\PYZlt{}\PYZhy{}} qqnorm\PY{p}{(}dsub\PY{o}{\PYZdl{}}age\PY{p}{,} plot \PY{o}{=} \PY{k+kc}{FALSE}\PY{p}{)}
                                   dsub\PY{o}{\PYZdl{}}xq \PY{o}{\PYZlt{}\PYZhy{}} \PY{k+kp}{q}\PY{o}{\PYZdl{}}x
                                   dsub
                               \PY{p}{\PYZcb{}}\PY{p}{)}
         
         age.qq \PY{o}{=} ggplot\PY{p}{(}data \PY{o}{=} cbMFQQ\PY{p}{,} aes\PY{p}{(}x \PY{o}{=} xq\PY{p}{,} y \PY{o}{=} age\PY{p}{,} color \PY{o}{=} year\PY{p}{)}\PY{p}{)} \PY{o}{+}
                         geom\PYZus{}point\PY{p}{(}alpha\PY{o}{=}\PY{l+m}{0.25}\PY{p}{)} \PY{o}{+}
                         geom\PYZus{}smooth\PY{p}{(}method \PY{o}{=} \PY{l+s}{\PYZdq{}}\PY{l+s}{lm\PYZdq{}}\PY{p}{,} se \PY{o}{=} \PY{k+kc}{FALSE}\PY{p}{)} \PY{o}{+}facet\PYZus{}wrap\PY{p}{(}\PY{o}{\PYZti{}}\PY{k+kp}{as.factor}\PY{p}{(}year\PY{p}{)}\PY{p}{,} nrow\PY{o}{=}\PY{l+m}{4}\PY{p}{)}\PY{o}{+}
                         xlab\PY{p}{(}\PY{l+s}{\PYZdq{}}\PY{l+s}{Normal Theoretical Quantiles\PYZdq{}}\PY{p}{)} \PY{o}{+}
                         ylab\PY{p}{(}\PY{l+s}{\PYZdq{}}\PY{l+s}{Normal Data Quantiles\PYZdq{}}\PY{p}{)}\PY{o}{+}
                         ggtitle\PY{p}{(}\PY{l+s}{\PYZdq{}}\PY{l+s}{QQ plot of Age by Year \PYZhy{} Female and Male\PYZdq{}}\PY{p}{)}
         age.qq
         \PY{k+kn}{detach}\PY{p}{(}package\PY{o}{:}plyr\PY{p}{)}
\end{Verbatim}


    
    
    \begin{center}
    \adjustimage{max size={0.9\linewidth}{0.9\paperheight}}{output_48_1.png}
    \end{center}
    { \hspace*{\fill} \\}
    
    \begin{itemize}
\tightlist
\item
  \textbf{boxplots analysis}
\end{itemize}

Density plots and QQ-plots do not give the complete picture of the
spread of the age distributions with a particular year. This can be
achieved using a Box-plot.

Side-by-Side boxplots are a great way to compare the spread of the age
distributions for all the years. From the plots below it is easy to
access the spread of the ages for each year broken down by gender.

Some key observations can be made from these plots. The age-range of
male runners is larger when compared with female runners, this trend
wasn't evident with earlier plots. Second, and perhaps an amusing
observation is that oldest female to participate in these events was in
2006, whereas the oldest male to participate in this event is in 2012,
both are outliers.

    \begin{Verbatim}[commandchars=\\\{\}]
{\color{incolor}In [{\color{incolor}68}]:} \PY{k+kp}{options}\PY{p}{(}repr.plot.width\PY{o}{=}\PY{l+m}{10}\PY{p}{,} repr.plot.height\PY{o}{=}\PY{l+m}{8}\PY{p}{)}
         bp.p \PY{o}{\PYZlt{}\PYZhy{}} ggplot\PY{p}{(}merged\PY{p}{,} aes\PY{p}{(}x\PY{o}{=}year\PY{p}{,} y\PY{o}{=}age\PY{p}{,} fill\PY{o}{=}year\PY{p}{)}\PY{p}{)} \PY{o}{+} geom\PYZus{}boxplot\PY{p}{(}\PY{p}{)} \PY{o}{+}facet\PYZus{}wrap\PY{p}{(}\PY{o}{\PYZti{}}\PY{k+kp}{as.factor}\PY{p}{(}sex\PY{p}{)}\PY{p}{,} nrow\PY{o}{=}\PY{l+m}{2}\PY{p}{)} \PY{o}{+}
         ggtitle\PY{p}{(}\PY{l+s}{\PYZdq{}}\PY{l+s}{Boxplots of Age distributions over the years broken down by gender\PYZdq{}}\PY{p}{)} \PY{o}{+}
         coord\PYZus{}flip\PY{p}{(}\PY{p}{)}
         bp.p
\end{Verbatim}


    
    
    \begin{center}
    \adjustimage{max size={0.9\linewidth}{0.9\paperheight}}{output_50_1.png}
    \end{center}
    { \hspace*{\fill} \\}
    
    A combined Box-plot shows that more and more young people are
participating in this race.

    \begin{Verbatim}[commandchars=\\\{\}]
{\color{incolor}In [{\color{incolor}69}]:} \PY{k+kp}{options}\PY{p}{(}repr.plot.width\PY{o}{=}\PY{l+m}{10}\PY{p}{,} repr.plot.height\PY{o}{=}\PY{l+m}{8}\PY{p}{)}
         bp.p \PY{o}{\PYZlt{}\PYZhy{}} ggplot\PY{p}{(}merged\PY{p}{,} aes\PY{p}{(}x\PY{o}{=}year\PY{p}{,} y\PY{o}{=}age\PY{p}{,} fill\PY{o}{=}year\PY{p}{)}\PY{p}{)} \PY{o}{+} geom\PYZus{}boxplot\PY{p}{(}\PY{p}{)} \PY{o}{+}
         ggtitle\PY{p}{(}\PY{l+s}{\PYZdq{}}\PY{l+s}{Boxplots of Age distributions over the years both genders combined\PYZdq{}}\PY{p}{)} \PY{o}{+}
         coord\PYZus{}flip\PY{p}{(}\PY{p}{)}
         bp.p
\end{Verbatim}


    
    
    \begin{center}
    \adjustimage{max size={0.9\linewidth}{0.9\paperheight}}{output_52_1.png}
    \end{center}
    { \hspace*{\fill} \\}
    
    \subsection{Conclusion and Future
Works}\label{conclusion-and-future-works}

    From our analysis of age distribution across all years from 1999 to
2012, the general population of both male and female race runners
display right-skewed distribution with the highest frequency of runners
falling into the 30-40 years age bin for Men and 20-30 age bin for
Women.

Using a stacked density plot to show a single curve for each year, there
is a decrease in mean age from 1999 to recent years for both sexes. Our
finding is supported by the average age line plot and box which shows
the age of men steadily decreases from a mean of \textasciitilde{}40 to
\textasciitilde{}37 while that of women decrease from a mean age of
\textasciitilde{}35 to \textasciitilde{}32 across the years. The drop in
mean age of men is quite significant compared to the drop in mean age of
women. The mean age of participating Men continues to be higher than the
mean of participating women from 1999 to 2012. The drop in age seems to
have plateud in 2008. To that front, we can speculate the decrease will
be slow moving forward, just as we have seen from 2008 to 2012.

Our future works could extend to this analysis. It would include
modeling performance metrics and age distributions using different
visualizations.

    \subsection{References}\label{references}

{[}1{]} D. Lang and D. Nolan, Data Science in R: A Case Studies Approach
to Computation Reasoning and Problem Solving. New York, New York: CRC
Press.


    % Add a bibliography block to the postdoc
    
    
    
    \end{document}
